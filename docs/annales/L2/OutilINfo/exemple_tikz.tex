\documentclass[11pt]{article}
\usepackage[french]{babel}
\usepackage[T1]{fontenc}
\usepackage{tikz}

\title{Illustration de l'utilisation de tikz}
\author{}
\date{}

\begin{document}

\maketitle

La figure~\ref{fig:expression} (gauche) est r\'ealis\'ee avec le package \texttt{tikz}. Il s'agit
de la repr\'e\-sentation, sous forme d'arbre, d'une expression arithm\'etique~; les n\oe{}uds entour\'es
de bleu sont les op\'erateurs de l'expression. Dans le code source, Les
n\oe{}uds sont positionn\'es sur la grille sous-jacente illustr\'ee sur la droite (l'origine du rep\`ere
\'etant le n\oe{}ud marqu\'e d'une fl\`eche bleue).

\begin{figure}[htpb]
  \begin{tikzpicture}%[scale=0.75]
    % Noeuds de l'arbre
    \node[circle, draw=blue] (racine) at (3,3) {$+$};
    \node[circle, draw=blue] (fils gauche) at (1,2) {$/$};
    \node (feuille 1) at (0,0) {$4$};
    \node (feuille 2) at (2,0) {$\sqrt{2}$};
    \node (feuille 3) at (5,2) {$3$};
    % Aretes
    \draw[->] (racine) -- (fils gauche);
    \draw[->] (fils gauche) -- (feuille 1);
    \draw[->] (fils gauche) -- node[right] {div. enti\`ere} (feuille 2);
    \draw[->] (racine) -- (feuille 3);
  \end{tikzpicture}
  \hfill
  \begin{tikzpicture}%[scale=0.75]
    % Grille
    \draw[dotted] (0,0) grid (5,3);
    % Coins
    \node[rectangle, draw=red] (coin 1) at (0,0) {$0,0$};
    \node[rectangle, draw=red] (coin 2) at (0,3) {$0,3$};
    \node[rectangle, draw=red] (coin 3) at (5,0) {$5,0$};
    \node[rectangle, draw=red] (coin 4) at (5,3) {$5,3$};
    % Coin special
    \coordinate (debut fleche) at (-1.5,0);
    \draw[->, thick, blue] (debut fleche) -- (coin 1);
    % Noeuds (sauf coins)
    \node (racine) at (3,3) {$3,3$};
    \node (fils gauche) at (1,2) {$1,2$};
    \node (feuille 2) at (2,0) {$2,0$};
    \node (feuille 3) at (5,2) {$5,2$};
    % Aretes
    \draw[dashed] (racine) -- (fils gauche);
    \draw[dashed] (fils gauche) -- (coin 1);
    \draw[dashed] (fils gauche) -- (feuille 2);
    \draw[dashed] (racine) -- (feuille 3);
  \end{tikzpicture}
  \caption{Une expression arithm\'etique (\`a gauche) et la construction (\`a droite)}
  \label{fig:expression}
\end{figure}

\end{document}
